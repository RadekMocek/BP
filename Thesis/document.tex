% !TEX TS-program = xelatex

% Inicializace tulthesis
\documentclass[FM]{tulthesis}
% Typografie pro češtinu; xelatex alternativa pro babel
\usepackage{polyglossia}
\setdefaultlanguage{czech}
% Automatické pevné mezery
\usepackage{xevlna}

% Pro seznam použité literatury
\usepackage[backend=biber, style=iso-numeric]{biblatex}
\addbibresource{zdroje.bib}

% Titulní strana
\TULtitle{Tvorba a využití botu pro výuku matematiky na platformě Discord}{Creating and using a bot for teaching mathematics on the Discord platform}
\TULprogramme{B0613A140005}{Informační technologie}{Information technology}
%\TULbranch{B0613A140005AI}{Aplikovaná informatika}{Applied Informatics}
\TULauthor{Radek Mocek}
\TULsupervisor{Ing.  Igor Kopetschke}
\TULyear{2024}

% Začátek dokumentu
\begin{document}
	% Prohlášení
	\ThesisStart{male}
	
	% Poděkování
	\begin{acknowledgement}
		Děkuji
	\end{acknowledgement}
	
	% Abstrakt česky
	\begin{abstractCZ}
		Abstrakt česky
	\end{abstractCZ}
	
	% Klíčová slova česky
	\begin{keywordsCZ}
		Klíčová slova česky
	\end{keywordsCZ}
	\vspace{2cm}
	
	% Abstrakt anglicky
	\begin{abstractEN}
		Abstrakt anglicky
	\end{abstractEN}
	
	% Klíčová slova anglicky
	\begin{keywordsEN}
		Klíčová slova anglicky
	\end{keywordsEN}
	
	% Obsah
	\tableofcontents
	
	% Seznam obrázků
	\listoffigures
	
	% Seznam tabulek
	%\listoftables
	
	\clearpage
	
	% Zkratky
	\begin{abbrList}
		\textbf{API} & application programming interface \\
		\textbf{VoIP} & Voice over Internet Protocol \\
	\end{abbrList}
	
	% Začátek hlavního textu
	
	% 1. Vypracujte rešerši sociálních platforem, které umožňují integraci botů.
	% 2. Analyzujte vybranou skupinu existujících botů na platformě Discord a knihoven pro jejich tvorbu.
	% 3. Navrhněte bot zaměřený na výklad a příklady z lineární algebry při využití specifických funkcí Discordu včetně administrace a interaktivních zpráv.
	% 4. Navržené řešení implementujte a nasaďte do testovacího provozu pro vybranou skupinu uživatelů.
	% 5. Vyhodnoďte zpětnou vazbu od uživatelů a navrhněte případné úpravy a vylepšení.
	
	\chapter{Úvod}
	
	Úvod.
	
	\chapter{Boti na sociálních platformách}
	
	Discord není jediná sociální platforma, která přímo podporuje tvorbu a integraci botů. Tato kapitola představuje některé další platformy, které disponují podobnými nástroji. Nejdříve je ale nutné upřesnit, jak jsou v tomto kontextu myšleny pojmy \textit{sociální platforma} a \textit{bot}.
	
	Discord je vcelku obtížné zařadit do jedné konkrétní kategorie softwaru. Dokud se nově zaregistrovaný uživatel nepřipojí na žádný server, pak se Discord chová jako VoIP a instant messaging aplikace, kde lze komunikovat pouze s lidmi, které si uživatel přidá do přátel. Po připojení na nějaký veřejný Discord server se ale aplikace přibližuje ke kategorii sociálních médií, kdy uživatel může sdílet a konzumovat obsah v rámci moderované komunity. Pojem sociální platforma je zde tedy myšlen spíše jako zastřešující termín pro Discord a jemu podobné aplikace, u nichž o zařazení do této kapitoly především rozhodovalo, zdali nějakým způsobem podporují integraci botů.
	
	Uživatelé sociálních médií často vnímají pojem bot negativně, jelikož se mezi boty mimo jiné řadí i programy generující spam nebo umělou návštěvnost za účelem zkreslení statistik \cite{lit3}. Na platformě Discord je ale bot speciální typ uživatele, jehož chování je automatizované a určené programem. S tímto botem lze zahájit přímá konverzace nebo může být pozván na server, kde pak pomocí volání Discord API dokáže provádět stejné akce jako běžný uživatel. Jedná se například o odesílání zpráv a čtení jejich obsahu, moderaci členů serveru, nebo i připojení do hlasového hovoru. Nemusí se jednat o chatbota, který se snaží imitovat lidské chování. Bot obvykle reaguje na předem danou sadu příkazů vykonáním nějaké činnosti.
	
	\section{Slack}
	
	Z populárních a hojně používaných platforem se Discordu vzhledově i funkčně nejvíce podobá program Slack. Ten byl spuštěn již v roce 2013 a od té doby si vybudoval reputaci jako standard pro komunikaci v technologických společnostech. \cite{lit3}
	
	Slack se prezentuje jako software usnadňující interní komunikaci v nějaké organizaci a dal by se nazvat jako \quotedblbase Discord pro firmy\textquotedblleft\,. Tomu odpovídá i zdejší výběr tzv. aplikací (\textit{Apps}), které se podobají zkoumaným botům. Tyto aplikace %typy obvyklých apps; jak je tvořit; pricing
	
	% Legacy boti https://api.slack.com/legacy/enabling-bot-users#limitations https://api.slack.com/legacy/custom-integrations/bot-users
	
	% https://revolt.chat/
	% https://element.io/
	% https://www.guilded.gg/
	% Teamspeak ?
	% https://www.proofhub.com/articles/discord-alternatives etc
	% Top sociální platformy
	
	\chapter{Boti na platformě Discord}
	
	% Zmínit obecně populární boty/jejich kategorie
	% Pak boti na matematiku
	% Pak prostředky pro jejich tvorbu (nebo to až v další kapitole a tuto sloučit s tou předchozí ?)
		
	\chapter{Tvorba Discord bota}
	
	\section{Návrh}
	
	\section{Implementace}
	
	\section{Zhodnocení}
	
	\chapter{Závěr}
	
	Závěr.
	
	% Zdroje
	\chapter*{Seznam použité literatury}
	\addcontentsline{toc}{chapter}{Seznam použité literatury}
	\printbibliography[heading=none]
	
\end{document}