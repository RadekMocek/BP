% !TEX TS-program = xelatex
\documentclass[FM]{tulpresentation}
\usepackage{pifont}
\title{Tvorba a využití botu pro výuku matematiky na platformě Discord}
\author{Radek Mocek}
\newcommand{\authorMail}{Ing. Igor Kopetschke}
\newcommand{\authorPhone}{}
\newcommand{\OO}{\item[$\square$]}
\newcommand{\XX}{\item[\rlap{\raisebox{0.3ex}{\hspace{0.4ex}\ding{52}}}$\square$]}
\begin{document}
	\TULtitleframe
	\begin{frame}
		\frametitle{Cíle bakalářské práce}
		\begin{enumerate}
			\item Vypracujte rešerši sociálních platforem, které umožňují integraci botů.
			\item Analyzujte vybranou skupinu existujících botů na platformě Discord a knihoven pro jejich tvorbu.
			\item Navrhněte bot zaměřený na výklad a příklady z lineární algebry při využití specifických funkcí Discordu včetně administrace a interaktivních zpráv.
			\item Navržené řešení implementujte a nasaďte do testovacího provozu pro vybranou skupinu uživatelů.
			\item Vyhodnoťte zpětnou vazbu od uživatelů a navrhněte případné úpravy a vylepšení.
		\end{enumerate}
	\end{frame}
	\begin{frame}
		\frametitle{Aktuální stav bakalářské práce}
		\bigskip
		\textbf{Praktická část}
		\begin{itemize}
			\XX Implementovat bota v discord.py
			\XX Nasadit bota do testovacího provozu
			\OO Posbírat a vyhodnotit zpětnou vazbu
		\end{itemize}
		\bigskip
		\textbf{Textová část}
		\begin{itemize}
			\XX Rešerše
			\OO Návrh, implementace, zhodnocení
			\OO Úvod, závěr, abstrakt
		\end{itemize}
	\end{frame}
\end{document}